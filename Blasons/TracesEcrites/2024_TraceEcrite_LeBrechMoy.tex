\documentclass[40pt]{article}
\usepackage[utf8]{inputenc}
\usepackage[french]{babel}
\usepackage{geometry}
 \geometry{
 a4paper,
 total={170mm,257mm},
 left=20mm,
 top=20mm,
 }
\usepackage[T1]{fontenc}
\usepackage{tgbonum}
\usepackage{setspace}
\onehalfspacing

\begin{document}
{\fontfamily{phv}\selectfont
\section*{Les Blasons}

\large Durant cette activité, vous avez décrit des motifs représentés sur des blasons à vos camarades, et ils ont reproduit le dessin à partir de vos instructions, et vous avez dessiné les dessins décrits par vos camarades, avec plus ou moins de précision.

C'est de l'informatique parce que comme vous, l'ordinateur n'est pas capable de comprendre des informations qui ne sont pas précises. Il faut d'abord se mettre d'accord sur ce que veut dire une instruction pour qu'il puisse faire exactement ce que l'on veut. C'est donc très utile pour créer des langages de programmation.

\section*{Les Blasons}

Durant cette activité, vous avez décrit des motifs représentés sur des blasons à vos camarades, et ils ont reproduit le dessin à partir de vos instructions, et vous avez dessiné les dessins décrits par vos camarades, avec plus ou moins de précision.

C'est de l'informatique parce que comme vous, l'ordinateur n'est pas capable de comprendre des informations qui ne sont pas précises. Il faut d'abord se mettre d'accord sur ce que veut dire une instruction pour qu'il puisse faire exactement ce que l'on veut. C'est donc très utile pour créer des langages de programmation.

\section*{Les Blasons}

\large Durant cette activité, vous avez décrit des motifs représentés sur des blasons à vos camarades, et ils ont reproduit le dessin à partir de vos instructions, et vous avez dessiné les dessins décrits par vos camarades, avec plus ou moins de précision.

C'est de l'informatique parce que comme vous, l'ordinateur n'est pas capable de comprendre des informations qui ne sont pas précises. Il faut d'abord se mettre d'accord sur ce que veut dire une instruction pour qu'il puisse faire exactement ce que l'on veut. C'est donc très utile pour créer des langages de programmation.

\section*{Les Blasons}

Durant cette activité, vous avez décrit des motifs représentés sur des blasons à vos camarades, et ils ont reproduit le dessin à partir de vos instructions, et vous avez dessiné les dessins décrits par vos camarades, avec plus ou moins de précision.

C'est de l'informatique parce que comme vous, l'ordinateur n'est pas capable de comprendre des informations qui ne sont pas précises. Il faut d'abord se mettre d'accord sur ce que veut dire une instruction pour qu'il puisse faire exactement ce que l'on veut. C'est donc très utile pour créer des langages de programmation.

\end{document}
