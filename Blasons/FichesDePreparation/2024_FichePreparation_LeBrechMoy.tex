\documentclass[10pt]{article}
\usepackage[utf8]{inputenc}
\usepackage[french]{babel}
\usepackage[T1]{fontenc}
\usepackage{mathrsfs}

\usepackage[a4paper,width=280mm,top=10mm,bottom=20mm, landscape]{geometry}

\usepackage{makecell}
\usepackage{booktabs}
\usepackage{array}
\usepackage{subfig}
\usepackage{amsmath,amsfonts,amssymb}
\usepackage{tikz}
\usepackage{tikz-cd}
\usetikzlibrary{shapes}
\usepackage{mathtools}
\usepackage{titlesec} 
\usepackage{enumitem}
\usepackage{listings}
\usepackage{mathdots}
\usepackage{cancel}
\usepackage{comment}

\usepackage{makeidx}

\usepackage{tgpagella}
\makeindex

\usepackage[utf8]{inputenc}

\usepackage{subfiles}

\renewcommand{\arraystretch}{1.5}


\renewcommand\theadalign{c}
\renewcommand\theadfont{\textfont}
\renewcommand\theadgape{\Gape[4pt]}
\renewcommand\cellgape{\Gape[4pt]}

\title{Fiches de préparation}

\author{Alan Le Brech, Clara Moy}
\date{}

\begin{document}
\section*{Fiche de préparation}
\vspace{-0.5cm}
\begin{center}
\begin{tabular}{|c|>{\hsize=0.3\hsize\centering\arraybackslash}c|c|c|c|}
\hline 
\multicolumn{5}{|c|}{
Alan Le Brech \& Clara Moy} \\
\hline 
\multicolumn{1}{|l|}{\textbf{Classe} : CM1/CM2} & \multicolumn{3}{|l|}{\textbf{Titre} : Les blasons.} & \multicolumn{1}{|l|}{\textbf{Séance n°1}} \\ 
\hline
\multicolumn{5}{|l|}{\textbf{Compétences travaillées} : Communication, élaboration d'instructions non ambigües} \\
\multicolumn{5}{|l|}{\textbf{Objectifs} : Faire comprendre l'intérêt d'avoir une manière de communiquer non ambigüe pour la programmation} \\
\hline
\textbf{Activité} & \textbf{Script} & \textbf{Notes/Objectifs} & \textbf{Matériel} & \textbf{Temps (min)} \\
\hline 
\thead{Présentation\\de l'activité} & \thead{
    - se présenter à nouveau \\
    - leur dire de mettre des petits chevalets avec leurs noms\\
    - leur montrer un modèle en leur expliquant qu'ils devront les décrire,\\ seulement avec des mots \\
    - leur montrer les blasons vide et leur expliquer qu'ils\\ devront suivre les instructions de celui qui décrit\\
    - faire un exemple, faire reformuler
    } & \thead{Insister sur le fait de pas regarder \\les feuilles des autres} & \thead{Tableau pour exemple} & \thead{7} \\ 
\hline 
\thead{Distribution\\du matériel} & \thead{
    - les dispatcher en groupes de 4 ou 3\\
    - demander à un élève de distribuer une feuilles A par groupe\\ 
    - demander à un autre de distribuer une feuille vierge par personne
    } & \thead{faire que ça soit rapide} & \thead{- feuilles A\\ - feuilles vierges} & \thead{2} \\ 
\hline 
\thead{Activité} & \thead{
    - passer dans les rangs pour voir si tout se passe bien
} & \thead{faire attention au volume sonore} & \thead{} & \thead{10} \\ 
\hline 
\thead{Remise en commun} & \thead{
    - reprendre l'attention\\
    - demander si ça s'est bien passé, les difficultés rencontrées\\
    - expliquer qu'on va leur distribuer de nouvelles \\feuilles avec des exemples de vocabulaire\\ 
    - dire de demander des feuilles vierges quand la leur est pleine
} &  \thead{qu'ils comprennent qu'il faut \\des instructions claires et précises} &\thead{} & \thead{5} \\ 
\hline 
\thead{Distribution\\du matériel} & \thead{
    - demander à un élève de distribuer une feuilles B par groupe\\ 
    - demander à un élève de distribuer une feuilles vocabulaire par groupe
    } & \thead{faire que ça soit rapide} & \thead{- feuilles B\\ - feuilles de vocabulaire\\ - feuilles vierges} & \thead{2} \\ 
\hline 
\thead{Remise au travail} & \thead{
    - passer dans les rangs pour voir si tout se passe bien\\
    - si ils sont en avance, leur donner la feuille C
} & \thead{faire attention au volume sonore} & \thead{- feuilles C} & \thead{19} \\ 
\hline 
\thead{Conclusion} & \thead{
    - leur demander leur avis\\
    - leur demander si ils savent en quoi c'est de l'informatique\\
    - leur expliquer en quoi c'est de l'informatique\\
    - faire reformuler\\
    - distribuer la trace écrite (si on en a une)
} & \thead{être clair et qu'il comprennent} & \thead{- trace écrite (si on en a fait)} & \thead{10} \\ 
\hline 
\multicolumn{4}{|l|}{} &
\multicolumn{1}{|c|}{\textbf{Total :} 55} \\
\hline
\end{tabular}
\end{center}

\textbf{Institutionnalisation}\\
    C'est de l'informatique parce que comme vous l'ordinateur il est pas capable de comprendre des informations qui sont pas précises. Il faut d'abord se mettre d'accord sur ce que veux dire une instruction pour qu'il puisse faire exactement ce que l'on veut. C'est comme ce que vous avez fait avec les blasons : parfois, vous disiez des choses qui étaient pas assez précises et à cause de ça, les autres faisaient des choses auxquelles vous ne vous attendiez pas.

\end{document}

