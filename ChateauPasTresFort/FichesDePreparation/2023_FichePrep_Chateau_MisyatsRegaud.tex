\documentclass[10pt]{article}
\usepackage[utf8]{inputenc}
\usepackage[french]{babel}
\usepackage[T1]{fontenc}
\usepackage{mathrsfs}

\usepackage[a4paper,width=280mm,top=10mm,bottom=20mm, landscape]{geometry}

\usepackage{makecell}
\usepackage{booktabs}
\usepackage{array}
\usepackage{subfig}
\usepackage{amsmath,amsfonts,amssymb}
\usepackage{tikz}
\usepackage{tikz-cd}
\usetikzlibrary{shapes}
\usepackage{mathtools}
\usepackage{titlesec} 
\usepackage{enumitem}
\usepackage{listings}
\usepackage{mathdots}
\usepackage{cancel}
\usepackage{comment}

\usepackage{makeidx}

\usepackage{tgpagella}
\makeindex

\usepackage[utf8]{inputenc}

\usepackage{subfiles}

\renewcommand{\arraystretch}{1.5}


\renewcommand\theadalign{c}
\renewcommand\theadfont{\textfont}
\renewcommand\theadgape{\Gape[4pt]}
\renewcommand\cellgape{\Gape[4pt]}

\begin{document}

\begin{center}
\begin{tabular}{|c|>{\hsize=0.3\hsize\centering\arraybackslash}c|c|c|c|}
\hline 
\multicolumn{5}{|c|}{
Misyats Nazar \& Regaud Gaetan} \\
\hline 
\multicolumn{1}{|l|}{\textbf{Classe} : CM2} & \multicolumn{3}{|l|}{\textbf{Titre} : Le château pas très fort} & \multicolumn{1}{|l|}{\textbf{Séance n°4}} \\ 
\hline
\multicolumn{5}{|l|}{\textbf{Compétences travaillées} : Vérification, raisonnement, logique, langage.} \\
\multicolumn{5}{|l|}{\textbf{Objectifs} : Introduire la vérification de formules dans un automate.} \\
\hline
\textbf{Activité} & \textbf{Script} & \textbf{Notes/Objectifs} & \textbf{Matériel} & \textbf{Temps} \\
\hline 
\thead{Préparation\\de la classe} & \thead{Temps de marge pour préparer les îlots etc.} & \thead{A faire en amont du début de classe.} & \thead{Fiches} & \thead{-} \\ 
\hline 
\thead{Présentation\\de l'activité} & \thead{Introduction devant la classe entière.} & \thead{Présentation avec un nombre restreint de cartes} & \thead{} & \thead{5} \\ 
\hline 
\thead{Explication\\avec exemple} & \thead{Les faire reformuler 4 fois.} & \thead{Pratique avec quelques exemples.} & \thead{} & \thead{3} \\ 
\hline 
\thead{Activité} & \thead{Écrire des phrases avec les cartes} & \thead{Compréhension des cartes} & \thead{} & \thead{5} \\ 
\hline 
\thead{Remise en commun} & \thead{Élèves $\hookrightarrow$ tableau pour faire exemple.\\ Ajout d'une carte + distribution château.} &  \thead{} &\thead{ } & \thead{5} \\ 
\hline 
\thead{Remise au travail} & \thead{Transcrire phrases avec cartes, décider si elles sont réalisables\\dans le château. Ajout progressif de cartes selon les groupes.\\ Algorithme de vérification efficace.} & \thead{} & \thead{} & \thead{20} \\ 
\hline 
\thead{Conclusion} & \thead{Bilan et trace écrite.} & \thead{Explications récapitulatives.} & \thead{ } & \thead{10} \\ 
\hline 
\thead{C'est de\\ l'informatique car\dots} & \thead{Vérification de propriétés dans des graphes,\\ systèmes avec des millions de paramètres,\\ vérifier si tout va bien : avion, voiture} & \thead{} & \thead{} & \thead{5} \\ 
\hline
\multicolumn{4}{|l|}{} &
\multicolumn{1}{|c|}{\textbf{Total :} 55} \\
\hline
\end{tabular}

\end{center}
\newpage

\end{document}