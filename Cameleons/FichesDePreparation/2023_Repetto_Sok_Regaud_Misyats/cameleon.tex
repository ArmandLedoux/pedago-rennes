\documentclass[main.tex]{subfiles}
\begin{document}
\begin{center}
\begin{tabular}{|c|>{\hsize=0.3\hsize\centering\arraybackslash}c|c|c|c|}
\hline
\multicolumn{5}{|c|}{
Repetto Guilhem \& Gaëtan Regaud \& Sok Chanattan \& Misyats Nazar} \\
\hline
\multicolumn{1}{|l|}{\textbf{Classe} : CM2} & \multicolumn{3}{|l|}{\textbf{Titre} : Les nids de caméléons.} & \multicolumn{1}{|l|}{\textbf{Séance n°1}} \\
\hline
\multicolumn{5}{|l|}{\textbf{Compétences travaillées} : Logique propositionnelle.} \\
\multicolumn{5}{|l|}{\textbf{Objectifs} : Introduire le raisonnement logique et les circuits informatique.} \\
\hline
\textbf{Activité} & \textbf{Script} & \textbf{Notes/Objectifs} & \textbf{Matériel} & \textbf{Temps (min)} \\
\hline
\thead{Préparation\\de la classe} & \thead{Temps de marge pour préparer les îlots etc.} & \thead{A faire en amont du début de classe.} & \thead{} & \thead{-} \\
\hline
\thead{Présentation\\de l'activité} & \thead{Introduction devant la classe entière.} & \thead{On commence avec des arbres de hauteur 2.} & \thead{} & \thead{5} \\
\hline
\thead{Explication\\avec exemple} & \thead{Les faire reformuler 4 fois} & \thead{Pratique avec quelques exemples.} & \thead{Feuilles des arbres et caméléons.} & \thead{3} \\
\hline
\thead{Distribution\\du matériel} & \thead{Parallélisable avec le second binôme\\pendant les explications.} & \thead{} & \thead{Feuilles des arbres et caméléons.} & \thead{2} \\
\hline
\thead{Activité} & \thead{Extension possible avec\\les arbres de hauteur 3.} & \thead{Chaque ilot traite des arbres de base\\différents à résoudre.} & \thead{Nouvelles feuilles d'arbres.} & \thead{15} \\
\hline
\thead{Remise en commun} & \thead{Bilan sur la 1$^{\text{ère}}$ partie\\et la solution qui marche à tous les coups.} &  \thead{Attirer l'attention de tous les ilots\\et demander à des élèves d'expliquer\\et d'expliquer aux autres.} &\thead{•} & \thead{5} \\
\hline
\thead{Remise au travail} & \thead{Extension avec les formules\\insatisfiables et oiseaux.} & \thead{} & \thead{} & \thead{15} \\
\hline
\thead{Conclusion} & \thead{Bilan et trace écrite\\+ illustration avec Logisim.} & \thead{Explications récapitulatives.} & \thead{Feuilles à coller pour les élèves.} & \thead{10} \\
\hline
\thead{C'est de l'informatique\\car\dots} & \thead{C'est lié aux portes logiques qui\\composent les circuits utilisés\\ dans les ordinateurs.} & \thead{Attirer l'attention des élèves\\les intéresser aux approfondissements.\\démonstration graphique au tableau\\avec logisim d'un circuit fonctionnel.} & \thead{} & \thead{5} \\
\hline
\multicolumn{4}{|l|}{} &
\multicolumn{1}{|c|}{\textbf{Total :} 55} \\
\hline
\end{tabular}

\end{center}

C'est de l'informatique car ce que vous avez fait c'est de la logique ! Les règles que vous avez utilisées
pour décider comment un caméléon change de couleur selon la couleur des autres caméléons
c'est des opérations logiques: ET, OU, NON.
[montrer schémas et expliquer les portes]
Tout ce que fait l'ordinateur il le fait avec la logique. En fait l'ordinateur c'est plein de circuits qui réalisent des opérations logiques, non pas avec des caméléons, mais avec du courant électrique !
Croyez moi quand je vous dis que ceci est un circuit [montrer circuit additionneur] qui additionne deux nombres
en ne faisant que de changer des couleurs de caméléons avec les règles que vous avez utilisées.

\end{document}
