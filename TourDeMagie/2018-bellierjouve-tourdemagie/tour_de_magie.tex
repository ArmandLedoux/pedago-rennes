\documentclass{article}

\usepackage{fullpage}
\usepackage[french]{babel}
\usepackage[T1]{fontenc}
\usepackage[utf8]{inputenc}
\usepackage{tabulary}

\title{Fiche de Préparation de l'activité sur les codes correcteurs}
\author{Jean Jouve \and Dylan Bellier}

\begin{document}

\maketitle

\newcommand{\columntitle}[1]{\textbf{#1}}

\section{Résumé de l'activité}

\subsection{Objectif}

Comprendre le principe de détection et de correction d'erreurs lors de la
transmission d'une information numérique.

\subsection{Compétences}

\begin{itemize}
    \item Utiliser ses connaissances pour traiter des problèmes
    \item Mettre en œuvre un raisonnement, articuler les différentes
      étapes d'une solution
\end{itemize}

\subsection{Niveaux}

CE2 et plus

\subsection{Durée}

45 minutes.

\subsection{Pré requis}

Nombres pairs et impairs; distinguer lignes et colonnes.

%On pose un ensemble de carte ayant deux couleurs différentes sur chaque face.
%On place ces cartes aléatoirement dans un carré dont chaque côté fait cinq
%cartes. Le magicien regarde ces cartes et s'en vas.
%On ajoute à chaque colonne et ligne une carte de sorte que chaque ligne et
%chaque colonne contiennent un nombre pair de cartes de chaque couleur.
%Puis on place un trésor, ici représenté par une carte plus petite que les
%autres, en dessous d'une de ces cartes que l'on retourne. Ainsi le magicien
%peut retrouver le trésor dans le carré de cartes en cherchant la ligne et la
%colonne ayant un nombre impair de cartes d'une même couleur.
%
%En expliquant la technique du magicien pour trouver le trésor aux élèves 
%ainsi qu'ils
%peuvent interpréter les cartes comme de l'information et le placement du
%trésor comme une erreur introduite dans cette information, on peut faire
%comprendre comment fonctionnent les codes correcteurs. 

\section{Déroulement}

\setlength{\tymin}{40pt}

\begin{center}
  \begin{tabulary}{\textwidth}{|L|L|L|L|L|}
    \hline
      \columntitle{Durée}
      & \columntitle{Phases}
      & \columntitle{Activité et consignes}
      & \columntitle{Organisation}
      & \columntitle{Matériel}
      \\
    \hline
    \hline
      5'
      & Préparation de l'activité
      & Si possible, avant que les élèves arrivent en classe, après avoir placé
        les tables en ilots de quatre places, nous plaçons des carrées de
        quatre sur quatre cartes sur chaque ilots.
      & Intervenants seul
      & Cartes
      \\
    \hline
      5'
      & Présentation de l'activité
      & Nous nous présentons, puis nous indiquons que nous allons faire un
        tour de magie qui a un lien avec l'informatique et que les élèves
        pourront comprendre le tour à la fin de la séance et aussi apprendre
        quelque chose sur l'informatique.
      & Collectif
      & Rien
      \\
    \hline
      10'
      & Démonstration du tour de magie
      & Nous indiquons aux élèves qu'ils peuvent retourner les cartes comme ils
        le veulent.
        
        Après avoir regardé les carrées de cartes de chaque
        ilot, une personne sort de la classe (elle prendra le rôle de
        magicien).
        
        Le reste des intervenants passe par chaque ilot pour placer
        des cartes à la fin de chaque
        ligne et colonne de sorte que ces lignes et colonnes contiennent un
        nombre pair de cartes vertes, ils indiqueront que ces cartes servent
        à rendre plus difficile le travail du magicien mais n'indiqueront pas
        aux élèves la façon dont ils placent les cartes. Ils demandent ensuite
        aux élèves de cacher un trésor en dessous d'une carte et de la
        retourner.

        Le magicien est appelé par les intervenants restées à l'intérieur de la
        salle, il passe ensuite sur chaque ilot et retrouve le trésor placé
        sous les cartes.
      & Groupes de 4 ou 5
      & Cartes
      \\
    \hline
      10'
      & Lancement de la réflexion
      & Nous expliquons aux élèves que les assistants aident le magicien au
        lieu de le gêner. Nous leur demandons de donner des idées sur la
        façon dont les assistants auraient pu aider le magicien. Si la bonne
        réponse est donnée nous expliquons à toute la classe comment fonctionne
        le tour de magie. Si des mauvaise réponse sont données, nous expliquons
        aux élèves pourquoi cela ne peut être une bonne réponse. Après une
        petite dizaine de minutes, nous expliquons le tour de magie à la
        classe.
      & Collectif
      & Cartes
      \\
    \hline
      10'
      & Pratique du tour de magie (facultatif)
      & Les élèves essaient de faire le tour de magie avec leurs camarades. Les
        intervenants passent par les différents ilots pour aider les élèves qui
        n'auraient éventuellement pas compris, et pour répondre au
        différentes questions que pourrait avoir les élèves.
      & Groupe de de 4 ou 5.
      & Cartes
      \\
    \hline
      5'
      & Conclusion
      & Nous expliquons aux élèves que la logique du tour de magie
        est utilisée tout les jours autour d'eux, comme dans les disque compacte
        (CD) ainsi que tout autre système de stockage d'informations et d'envoie
        d'information. Nous expliquons ensuite les limites du système du tour de
        magie: on ne peut pas toujours corriger l'erreur, et on ne peut pas
        toujours la détecter.
      & Collectif
      & Rien
      \\
    \hline
  \end{tabulary}
\end{center}


\end{document}
