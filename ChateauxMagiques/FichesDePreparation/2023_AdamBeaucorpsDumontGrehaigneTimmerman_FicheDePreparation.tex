\documentclass[12pt,landscape]{article}
\usepackage[a3paper, margin=0.5cm]{geometry}

\usepackage[french]{babel}
\usepackage[T1]{fontenc}

\author{Paul ADAM \and Romain de Beaucorps \and Elie DUMONT \and Laura GREHAIGNE \and Jules TIMMERMAN}
\title{Fiche de préparation :\\Châteaux Magiques}

\begin{document}

\begin{tabular}{|c|c|c|c|c|}
	\hline
	\textbf{Titre} & \multicolumn{4}{|c|}{Château Magique}\\
	\hline
	% TODO : TEMPS
	\textbf{Durée} & \multicolumn{4}{|c|}{45 minutes}\\
	\hline
	\textbf{Prérequis} & \multicolumn{4}{|c|}{
		Suivre un chemin
	}\\
	\hline
	\textbf{Description} & \multicolumn{4}{|c|}{
		L'objectif est le parcours dans un automate et un graphe orienté pour trouver des chemins vérifiant certaines propriétés.
	}\\
	\hline
	\textbf{Matériel} & \multicolumn{4}{|c|}{
		Carte du château (automate)
	}\\
	\hline
	\textbf{Durée} & \textbf{Phase} & \textbf{Activités} & \textbf{Orga} & \textbf{Matériel}\\
	\hline

	2' & Présentation & Vous devez aider le magicien à trouver la Salle du trône ! Donnez-lui la formule magique qui lui permettra d'ouvrir les couloirs du château. & Tableau & Tableau\\
	& & Nous allons donner une fiche avec des questions et la carte du château & &\\

	\hline

	3' & Exemple & Exemple de déplacement (dans un automate) dessiné à la main & Tableau & Feutre\\
	
	\hline

	10' & Travail & Remplissage des questions. Pour les plus rapides, on en donne d'autres. & Groupe de 3 & Petite Carte et Questions\\

	\hline
	
	5' & Remise en commun & Quelle réponse pour telle question ? Plusieurs réponses possibles ? On part sur la version grande & Tableau & Tableau\\

	\hline

	10' & Travail & De nouveau avec le plus grand, nouvelles questions plus techniques. & Groupe de 3 & Grand Carte et Questions\\

	\hline

	10' & Remise en commun & idem & Tableau & Tableau\\

	\hline

	5' & InfoParceQue & ... & Tableau & Tableau\\

	\hline


	& \multicolumn{4}{|c|}{
		Le Château $\sim$ Automate = état d'un programme, Transition $\sim$ Condition
	}\\
	\textbf{Info parce que} & \multicolumn{4}{|c|}{
		Réfléchir sur le château, c'est comme réfléchir sur un programme, ce qu'il peut faire etc... (exemple avec l'accessibilité)
	}\\
	& \multicolumn{4}{|c|}{
		Automate aussi avec le langage, forme simple d'ordinateur
	}\\

	
	\hline
\end{tabular}


\end{document}