\documentclass[a4paper,11pt]{article}%

\usepackage{fullpage}%
\usepackage{url}%
\usepackage[T1]{fontenc}%
\usepackage[utf8]{inputenc}%
\usepackage[main=francais]{babel}% % Adjust the main language

\usepackage{graphicx}%

\title{Activité "Châteaux Magiques"}%
\author{Théo Degioanni \and Igor Martayan \and Arthur Dumas \and Matthieu Rodet}%
\date{}%

\begin{document}%

\maketitle%

\section{Résumé de l'activité}

\paragraph{Objectif}
Caractériser un ensemble de mots par un automate fini.

\url{https://fr.wikipedia.org/wiki/Automate_fini}

\paragraph{Niveau}
CM1 - CM2

\paragraph{Durée}
Environ 45 minutes.

\paragraph{Groupes}
Binômes puis groupes de 4 (fusion de binômes).

\section{Déroulement}

\begin{center}
\begin{tabular}{|l|p{5.5em}|p{15em}|p{6em}|p{6em}|}
  \hline
    \textbf{Durée} 
  & \textbf{Phase} 
  & \textbf{Activité et consignes} 
  & \textbf{Organisation} 
  & \textbf{Matériel} 
  \\
  \hline
  \hline
  <5' 
  & Présentation de l'activité 
% & Contexte : le roi a été enlevé, et les héros du royaume doivent le retrouver dans la salle du trône du château ennemi. Malheureusement, tous les couloirs du château sont protégés par des sortilèges, et les héros ont besoin de l'aide d'un magicien pour se déplacer entre les salles.
  & Contexte : Le roi a été enlevé et votre but est de le sauver. Il est enfermé dans la salle du trône, et des gardes se trouvent dans toutes les autres pièces. Notre héros, à l'aide d'un sorcier, doit donc se téléporter avec une formule magique directement dans la pièce du roi en suivant le plan magique du château.
  & Oral Collectif 
  & (Aucun)
  \\
  \hline
  5/10' 
  & Consignes 
  & Exemples de formules magiques pour se déplacer dans le château. Il faut se rendre du pont-levis à la salle du trône. Le magicien doit combiner des morceaux de formules pour décrire un chemin caractérisé par cette formule. 
  & Oral Collectif 
  & Un plan du petit château
  \\
  \hline
  10'/15'
  & Mise en pratique 1
  & Les élèves se répartissent en binômes : l'un d'entre eux est le magicien qui propose une formule, et l'autre est le héros qui suit le chemin indiqué morceau par morceau (les rôles tournent dans un groupe de trois). On échange les rôles une fois qu'un chemin gagnant est trouvé, pour former un autre chemin gagnant. Les encadrants circulent dans la salle pour vérifier que les consignes sont suivies et aiguiller les binômes qui peinent.
    
    Déroulement des questions en fonction de l'avancée des groupes :
    \begin{itemize}
      \item Trouver un sortilège qui mène à la salle du trône.
      \item Le magicien est maniaque et il ne veut se déplacer que 6 fois.
      \item Est-ce qu'il y a un sortilège le plus long ?
      \item Y a-t-il un endroit à éviter à tout prix ?
      \item Est-ce qu'il y a un sortilège qui commence par abracadabra et qui mène au trône ?
    \end{itemize}
  & En groupes de 2
  & Un petit château par groupe + une feuille réponse par personne
  \\
  \hline
\end{tabular}
\end{center}

\newpage

\begin{center}
\begin{tabular}{|l|p{5.5em}|p{15em}|p{6em}|p{6em}|}
  \hline
    \textbf{Durée} 
  & \textbf{Phase} 
  & \textbf{Activité et consignes} 
  & \textbf{Organisation} 
  & \textbf{Matériel} 
  \\
  \hline
  \hline
  5'
  & Mise en commun
  & On compare les mots correspondant aux chemins gagnants des différents groupes. Sont-ils tous les mêmes ? Se ressemblent-ils ? Quels sont leurs points communs ? 
  & Oral collectif
  & (Aucun)
  \\
  \hline
  10'
  & Mise en pratique 2
  & L'activité reprend dans un château plus grand et plus complexe. Les groupes à l'aise peuvent retourner travailler sur les façon de faire des chemins de longueur arbitrairement grande dans le petit château.

  Déroulement des questions en fonction de l'avancée des groupes :
  \begin{itemize}
    \item Trouver un sortilège qui mène à la salle du trône.
    \item Chemin sans passer par la salle du banquet (ou trésor).
    \item Y a-t-il un endroit à éviter à tout prix ?
    \item Quel est le mot le plus court ?
  \end{itemize}
  & En groupe de 4 (fusion de binômes)
  & Un grand château par groupe + une feuille réponse par personne
  \\
  \hline
  5'/10'
  & Mise en commun et conclusion
  & Comparaison des mots formés. Remarques sur la structure du château.
    Section "c'est de l'infomatique parce que...".
  & Oral collectif
  & (Aucun)
  \\
  \hline
\end{tabular}
\end{center}

\end{document} 