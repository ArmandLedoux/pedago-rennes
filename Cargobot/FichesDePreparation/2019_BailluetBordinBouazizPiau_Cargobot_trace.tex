
\documentclass[a4paper,11pt]{article}
 
\usepackage{fullpage}%
 
\usepackage[T1]{fontenc}%
\usepackage[utf8]{inputenc}%
 
\usepackage{mathpazo} 
 
\usepackage[main=francais,english]{babel}%  To be modified according to the language used
 
\usepackage{graphicx}%
 
\usepackage{url}%
\usepackage{abstract}%
 
\usepackage{minted}%


\usepackage{xcolor}
\definecolor{very-light-gray}{gray}{0.97}
 
\setminted{
  frame=lines,
  framesep=2mm,
  % baselinestretch=1.2,
  bgcolor=very-light-gray,
  % fontsize=\footnotesize,
  % linenos
}
 

%%%%%%%%%%%%%%%%%%%%%%%%%%%%%%%%%%%
 
\parskip=0.5\baselineskip%
 
\sloppy%
 
%%%%%%%%%%%%%%%%%%%%%%%%%%%%%%%%%%%%%%%%%%%%%%%%%%%%%%%%%%%%%%%%%%%%%%
\begin{document} 
 
\begin{center}
\huge
Cargo-bot
\end{center}

\textbf{Présentation~:}
Nous devons faire exécuter des ordres à un robot : une grue doit déplacer un ensemble de gobelets d’une situation initiale a une situation finale. Le but de l'activité est de créer un programme pour résoudre ces problèmes, et ceci en donnant un série d'instructions à la machine. La machine n'a pas à refléchir et se contente d'effectuer les instructions dans l'ordre.


\textbf{\`A retenir~:} 
Parler de l’importance de l’ordre des instructions.
Parler de la non-unicité d’un algorithme pour résoudre un problème.
Parler de l’utilité des structures conditionnelles.
Langage de programmations.

\end{document}
