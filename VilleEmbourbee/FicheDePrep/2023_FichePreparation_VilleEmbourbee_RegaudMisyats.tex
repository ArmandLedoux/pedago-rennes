\documentclass[10pt]{article}
\usepackage[utf8]{inputenc}
\usepackage[french]{babel}
\usepackage[T1]{fontenc}
\usepackage{mathrsfs}

\usepackage[a4paper,width=280mm,top=10mm,bottom=20mm, landscape]{geometry}

\usepackage{makecell}
\usepackage{booktabs}
\usepackage{array}
\usepackage{subfig}
\usepackage{amsmath,amsfonts,amssymb}
\usepackage{tikz}
\usepackage{tikz-cd}
\usetikzlibrary{shapes}
\usepackage{mathtools}
\usepackage{titlesec} 
\usepackage{enumitem}
\usepackage{listings}
\usepackage{mathdots}
\usepackage{cancel}
\usepackage{comment}

\usepackage{makeidx}

\usepackage{tgpagella}
\makeindex

\usepackage[utf8]{inputenc}

\usepackage{subfiles}

\renewcommand{\arraystretch}{1.5}


\renewcommand\theadalign{c}
\renewcommand\theadfont{\textfont}
\renewcommand\theadgape{\Gape[4pt]}
\renewcommand\cellgape{\Gape[4pt]}

\title{Fiche de préparation}

\author{Misyats Nazar, Regaud Gaëtan}
\date{}
\maketitle

\begin{document}
\begin{center}
\begin{tabular}{|c|>{\hsize=0.3\hsize\centering\arraybackslash}c|c|c|c|}
\hline 
\multicolumn{5}{|c|}{
Gaëtan Regaud \& Misyats Nazar} \\
\hline 
\multicolumn{1}{|l|}{\textbf{Classe} : CM2} & \multicolumn{3}{|l|}{\textbf{Titre} : La ville embourbée.} & \multicolumn{1}{|l|}{\textbf{Séance n°4}} \\ 
\hline
\multicolumn{5}{|l|}{\textbf{Compétences travaillées} : Communication, élaboration d'un vocabulaire.} \\
\multicolumn{5}{|l|}{\textbf{Objectifs} : } \\
\hline
\textbf{Activité} & \textbf{Script} & \textbf{Notes/Objectifs} & \textbf{Matériel} & \textbf{Temps (min)} \\
\hline 
\thead{Rappel de la fois précédente} & \thead{} & \thead{.} & \thead{} & \thead{5} \\ 
\hline 
\thead{Présentation\\de l'activité} & \thead{Introduction devant la classe entière \\+ histoire et consignes.} & \thead{} & \thead{} & \thead{5} \\ 
\hline 
\thead{Exemple et reformulation par un élève} & \thead{Passage au tableau ou animation puis reformulation} & \thead{Montrer le jeu en action pour \\comprendre les règles} & \thead{Toute petite ville\\Aimants pour paver la route} & \thead{5} \\ 
\hline 
\thead{Distribution du matériel} & \thead{donner la même première fiche} & \thead{} & \thead{Distribuer plusieurs feuilles\\ en fonction de l'avancée des élèves} & \thead{2} \\ 
\hline 
\thead{Activité} & \thead{Recherche par groupe de 2 ou 3 \\Extension en donnant des couleurs aux maisons\\(ne pas relier les voisins de même couleur)} & \thead{Trouver l'algo de Kruskal \\ comprendre la résistance\\aux changements} & \thead{} & \thead{20} \\ 
\hline 
\thead{Remise en commun} & \thead{Explication de l'algo de Kruskal avec un schéma} &  \thead{Expliquer l'algo de Kruskal} &\thead{Dessiner à la main ou avec \\ une animation} & \thead{10} \\ 
\hline 
\thead{Application de l'algo} & \thead{On leur donne un gros graphe et \\ils appliquent l'algo\\(ils garderont la feuille)} & \thead{Floyd-Warshall\\si ils ont été trop rapides} & \thead{} & \thead{5} \\ 
\hline 
\thead{C'est de l'informatique parce que ...} & \thead{Exemple des réseaux électriques, ... } & \thead{.} & \thead{} & \thead{8} \\ 
\hline 
\multicolumn{4}{|l|}{} &
\multicolumn{1}{|c|}{\textbf{Total :60 min} } \\
\hline
\end{tabular}
\end{center}
\end{document}
