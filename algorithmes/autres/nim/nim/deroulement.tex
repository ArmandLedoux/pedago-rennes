% Cette fiche d'activité à été développée par
% + Santiago Bautista <Santiago.Bautista@ens-rennes.fr>
% + Benjamin Bordais <Benjamin.Bordais@ens-rennes.fr>

% Elle est distribuée sous la licence Creative-Commons by-nc-sa 4.0
% (Attribution, Non-Commercial, Share Alike)
% qui peut être trouvée dans
% [le site de Creative Commons](https://creativecommons.org/licenses/by-nc-sa/4.0/)


\documentclass[12pt, a4paper]{article}
\usepackage[french]{babel}
\usepackage[utf8]{inputenc}
\usepackage[top=0.9in, left=0.6in, right=0.7in, bottom=1in]{geometry}
\usepackage{amssymb}
\usepackage{color}
\usepackage{fancyhdr}
\pagestyle{fancy}
\usepackage{longtable}
\usepackage{array}

\renewcommand{\headrulewidth}{1pt}
\chead{} 
\lhead{Jeu de Nim - Plan de l'activité}
\rhead{Santiago Bautista et Benjamin Bordais}

\renewcommand{\footrulewidth}{1pt}
\cfoot{\textbf{page \thepage}} 
\lfoot{2017}
\rfoot{ENS Rennes}

\fancypagestyle{plain}{
\fancyhf{} 
\cfoot{\textbf{page \thepage}} 
\renewcommand{\headrulewidth}{0pt}
\renewcommand{\footrulewidth}{1pt}}

\begin{document}
\title{Jeu de Nim}
\author{Santiago Bautista, Benjamin Bordais}
\date{\today}
\maketitle
\paragraph*{Niveau:} CM1-CM2
\paragraph*{Durée:} 45 minutes
\paragraph*{Prérequis:} Aucun
\paragraph*{Description du jeu:}
Dans le jeu de Nim, deux joueurs s'affrontent. Sur la table, entre eux, il y a 16 pions/jetons/allumettes
qui sont posés. A tour de rôle, chaque joueur prend un, deux ou trois jetons.
Le joueur qui prend le(s) derniers jetons a gagné.

\begin{longtable}{c|m{2.5cm}|m{5cm}|m{2.5cm}|m{3cm}}
\textbf{Durée} & \textbf{Phase} & \textbf{Activités et consignes} & \textbf{Organisation} & \textbf{Matériel} \\ \hline

5' &

Présentation des règles

 &

\og Je vous propose de jouer à un petit jeu. 

Nous allons vous montrer comment on y joue. \fg{}

On montre le déroulement d’une partie au tableau.

& 

(Santiago et Benjamin)
 
 &

Tableau et craies (ou feutres)


\\ \hline

5' &

Appropriation des règles
 
 &


On joue contre quatre élèves volontaires. Chacun de nous deux en rencontre deux. 
On fait remarquer aux élèves que l’on gagne à chaque fois.
  
& Idem & Idem \\ \hline

5' & 

Exploration du jeu

 &

Les élèves jouent l’un contre l’autre par groupes de deux
 
& 

Par groupes de deux
&
Une quinzaine de groupes de 16 jetons colorés


 \\ \hline


10'
&

Recherche de stratégie

  & 

\og Recommencez à jouer entre vous, mais maintenant l'objectif est de mettre en place une stratégie pour gagner.\fg{}
(leur rappeler que nous avons gagné chaque partie).

Aide des groupes en difficultés et défis supplémentaires pour les groupes en avance.

& Idem & Idem \\ \hline

10' &

Automatisation de la stratégie 
 
&  

Dans chacun des anciens groupes de deux, il y en a un qui est un \emph{robot} et un qui donne les instructions, il a les yeux fermés.
Ils alternent qui commence, ainsi que les rôles.


& 

Par groupes de quatre
 
 & Idem \\ \hline

5' & Démonstration & 
On choisit un élève à l’aise avec cet exercice pour faire une démonstration,
l’un de nous est le robot.
&
Oral collectif
&
Tableau (et craies ou feutres) \\  \hline

5' & En quoi est-ce de l'informatique &
\og 
Ce qu'on vient de faire, c'est de l'informatique
car on a besoin de stratégies gagnantes (que l'on appelle algorithmes)
 pour utiliser des ordinateurs.
\fg{}  
 
\og 
C'est que les ordinateurs sont très obéissants:
ils font tout ce qu'on leur demande, mais seulement ce que l'on demande.
\fg{} 

\og 
Il faut donc avoir une stratégie gagnante pour chaque problème que l'on veut faire résoudre à l'ordinateur,
afin de s'assurer qu'il résoudra le problème à coup sûr
\fg{} 

& Oral collectif & Rien
\end{longtable}

\paragraph*{Coup de pouce pour débloquer la réflexion} (si les élèves patinent) :
\begin{itemize}
\item Ranger les jetons/allumettes par couleurs, donc par groupes de 4
\item On peut jouer avec seulement 4 jetons, puis rajouter 4 autres jetons une fois qu'ils auront compris,
et ainsi de suite.
\item On joue contre l'élève
avec la consigne \og regarde bien ce que je fais\fg{}
puis on inverse les rôles en cours de partie.
\item On propose des stratégies (qui marchent pas) et on demande aux élèves de jouer contre.
\begin{itemize}
\item[\emph{Agressif}] prend toujours 3 pions (ou tous les pions s'il en reste moins), systématiquement.
\item[\emph{Peureux}] prend toujours un seul pion, systématiquement, quoi qu'il arrive.
\item[\emph{Aléatoire}] détermination au D6 du nombre de pions pris
\end{itemize}
\end{itemize}

\paragraph*{Variantes pour aller plus loin} (si certains groupes d'élèves vont trop vite)
\begin{itemize}
\item Est-ce que tu préfères commencer ou pas?
\item Et s'il y a 17 jetons au début, que se passe-t-il ?
\item Maintenant, celui qui prend le dernier jeton perd
\end{itemize}

\end{document}
