 
\documentclass[main.tex]{subfiles}
\begin{document}
\begin{tabular}{|c|>{\hsize=0.3\hsize\centering\arraybackslash}c|c|c|c|}
\hline
\multicolumn{5}{|c|}{
Repetto Guilhem \& Sok Chanattan} \\
\hline
\multicolumn{1}{|l|}{\textbf{Classe} : CM2} & \multicolumn{3}{|l|}{\textbf{Titre} : Les carrés magiques} & \multicolumn{1}{|l|}{\textbf{Séance n°4}} \\
\hline
\multicolumn{5}{|l|}{\textbf{Compétences travaillées} : Théorie de l'information et encodage.} \\
\multicolumn{5}{|l|}{\textbf{Objectifs} : Introduire la théorie de l'information et la déduction d'un encodage.} \\
\hline
\textbf{Activité} & \textbf{Script} & \textbf{Notes/Objectifs} & \textbf{Matériel} & \textbf{Temps} \\
\hline
\thead{Préparation\\de la classe} & \thead{Temps de marge pour préparer les îlots etc.} & \thead{A faire en amont du début de classe.} & \thead{} & \thead{-} \\
\hline
\thead{Présentation\\de l'activité} & \thead{Introduction devant la classe entière.} & \thead{On commence par présenter le\\problème vulgarisé en\\expliquant l'objectif.} & \thead{Carrés magiques} & \thead{5} \\
\hline
\thead{Explication\\avec exemple} & \thead{Les faire reformuler 4 fois.} & \thead{Pratique avec quelques exemples et démonstration magique.} & \thead{} & \thead{3} \\
\hline
\thead{Distribution\\du matériel} & \thead{Parallélisable\\pendant les explications.} & \thead{} & \thead{} & \thead{2} \\
\hline
\thead{Activité} & \thead{Laisser les élèves réfléchir et exhiber des idées.\\L'activité repose sur une solution.} & \thead{Chaque ilot peut simuler\\en extension le magicien\\l'assistant et le spectateur.} & \thead{} & \thead{15} \\
\hline
\thead{Remise en commun} & \thead{Bilan sur la 1$^{\text{ère}}$ partie\\et mise commun des idées.} &  \thead{Attirer l'attention de tous les ilots\\et demander à des élèves d'expliquer\\et d'expliquer aux autres.\\Possibilité d'appeler des élèves au tableau.\\
Extensions possibles : code barre,\\1 erreur pour un message de 3bits et\\la vraie magie (deviner le carré sans regarder).} &\thead{ } & \thead{5} \\
\hline
\thead{Remise au travail} & \thead{Introduction éventuelle d'extensions par ilot.} & \thead{} & \thead{} & \thead{15} \\
\hline
\thead{Conclusion} & \thead{Bilan et trace écrite.} & \thead{Explications récapitulatives.} & \thead{ } & \thead{10} \\
\hline
\thead{C'est de\\ l'informatique car\dots} & \thead{C'est un problème qui est très important\\en sécurité et en communication informatique,\\cela permet de garder l'information correcte\\face aux erreurs lors d'un échange\\de messages dans un réseau et de faire\\attention à bien protéger les informations\\dans les messages dans le cas d'attaques.} & \thead{} & \thead{} & \thead{5} \\
\hline
\multicolumn{4}{|l|}{} &
\multicolumn{1}{|c|}{\textbf{Total :} 55} \\
\hline
\end{tabular}

\end{document}
