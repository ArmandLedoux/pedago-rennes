\documentclass[a4paper,11pt]{article}%

\usepackage{fullpage}%
\usepackage{url}%
\usepackage[T1]{fontenc}%
\usepackage[utf8]{inputenc}%
\usepackage[main=francais]{babel}% % Adjust the main language

\usepackage{graphicx}%

\title{Activité "Algorithmes de Tri"}%
\author{Alexandre Drewery \and Yan Garito}%
\date{}%

\begin{document}%

\maketitle%

\section{Résumé de l'activité}

\paragraph{Objectif}
Découvrir la puissance d'un algorithme de tri classique.

\paragraph{Niveau}
CM1 - CM2

\paragraph{Durée}
Environ 45 minutes.

\paragraph{Groupes}
Binômes.

\paragraph{Remarque}
L'activité exploite le fait que la notion d'algorithme ait déjà été développée dans le cadre des premières activités des deux autres binômes.

\section{Déroulement}

\begin{center}
\begin{tabular}{|l|p{5.5em}|p{15em}|p{6em}|p{6em}|}
  \hline
    \textbf{Durée} 
  & \textbf{Phase} 
  & \textbf{Activité et consignes} 
  & \textbf{Organisation} 
  & \textbf{Matériel} 
  \\
  \hline
  \hline
  10'
  & Présentation et consignes
  & Les élèves vont devoir trier des cartes. Présentations de situations de la vie de tous les jours où un tri est utilisé. Cependant handicap : les cartes seront faces cachées. Explication des consignes.
  & Oral Collectif 
  & Un jeu de carte de démonstration
  \\
  \hline
  
  \hline
  10'/15'
  & Mise en pratique
  & Les élèves se répartissent en binômes. Les cartes sont mélangées et placées faces cachées sur la table. L'un des deux élèves doit les trier par ordre croissant. Pour ce faire il est libre de les arranger comme il le souhaite et peut demander à son camarade de comparer la valeur de deux cartes. Le comparateur (seul à pouvoir regarder les faces des cartes) indique alors quelle carte est la plus grande, sans lui donner les valeurs. Les rôles tournent après une réussite, ou si on n'avance pas.
  & En groupes de 2
  & 10 cartes numérotées à trier
  \\
  \hline
  < 5'
  & Mise en commun
  & Les élèves sont-ils parvenus à trier les cartes ? Si non, c'est normal. Si oui, était-ce simple ? Était-ce rapide ?
  & Oral collectif
  & (Aucun)
  \\
  \hline
  10'
  & Découverte d'un algorithme de tri
  & Chaque binôme reçoit un procédure de tri pour les guider vers une résolution du problème.
  & En groupes de 2
  & 10 cartes numérotées à trier et un algorithme de tri à essayer par groupe.
  \\
  \hline
  5'
  & Mise en commun et conclusion
  & L'algorithme était-il efficace ? Était-il rapide ?
    Section "c'est de l'infomatique parce que...".
  & Oral collectif
  & (Aucun)
  \\
  \hline
\end{tabular}
\end{center}

\end{document}%
