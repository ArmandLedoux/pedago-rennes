\documentclass[a4paper,11pt]{article}%

\usepackage{fullpage}%
\usepackage{url}%
\usepackage[T1]{fontenc}%
\usepackage[utf8]{inputenc}%
\usepackage[main=english]{babel}% % Adjust the main language

\usepackage{graphicx}%

\title{Activité "Algorithmes de Tri"}%
\author{Armand Ledoux \and Victor Jourdan}%
\date{}%

\begin{document}%

\maketitle%

\section{Résumé de l'activité}

\paragraph{Objectif}
Découvrir la notion d'algorithme à travers les algorithmes de tri classique puis une introduction à la dichotomie.

\paragraph{Niveau}
CM1 - CM2 - 6e

\paragraph{Durée}
Environ 50 minutes.

\paragraph{Groupes}
Binômes.



\section{Déroulement}

\begin{center}
\begin{tabular}{|l|p{5.5em}|p{15em}|p{6em}|p{6em}|}
  \hline
    \textbf{Durée} 
  & \textbf{Phase} 
  & \textbf{Activité et consignes} 
  & \textbf{Organisation} 
  & \textbf{Matériel} 
  \\
  \hline
  \hline
  3'
  & Installation
  & 
  & Par binôme 
  & Tables et chaises
  \\
  \hline
  5'
  & Présentation et consignes
  & Les élèves vont devoir trier des cartes. Cependant handicap : les cartes seront faces cachées. Explication des consignes de la première partie. Ne pas oublier de demander de reformuler les consignes.
  & Oral Collectif 
  & 10 cartes de démonstration
  \\
  \hline
  
  \hline
  15'
  & Mise en pratique (première partie)
  & Passer dans les rangs et les inciter à changer de rôles lorsqu'ils réussissent ou échouent. Tout le monde ne passera pas le même temps dans tous les rôles.
  & En groupes de 2
  & 10 cartes numérotées à trier
  \\
  \hline
  < 5'
  & Mise en commun
  & Les élèves sont-ils parvenus à trier les cartes ? Si non, c'est normal. Si oui, était-ce simple ? Était-ce rapide ? Présenter la deuxième partie. 
  & Oral collectif
  & (Aucun)
  \\
  \hline
  15'
  & Découverte d'un algorithme de tri (deuxième partie)
  & Chaque binôme reçoit un feuille contenant des algorithmes de tri à appliquer. Les inciter à échanger de rôle pour ceux qui ne l'ont pas trop fait. Même règles que la première partie.
  & En groupes de 2
  & 10 cartes numérotées à trier et une ou deux feuille d'algorithmes de tri par binôme.
  \\ 
  \hline
  2'
  & Mise en commun 
  & Cela était-il plus facile avec les algorithmes ? 
  & Oral collectif
  & (Aucun)
  \\
  \hline
  8' 
  & Activité dichotomie (troisième partie)
  & Explication et activité. L'objectif est qu'ils sentent que ce n'est que de la chance au début et qu'il faut trier les cartes. Leur expliquer ensuite la dichotomie et enchâiner sur la conclusion.
  & Sur une table visible au centre de la classe.
  & 7 cartes (pas forcément de même couleur et potentiellement avec des doublons)
  \\ 
  \hline
  7'
  & Conclusion
  & Quelques ordre de grandeurs impressionnants sur le nombre de coups qu'il aurait fallu pour trouver une carte selon le nombre de cartes de la troisième partie (4 coups pour 15, 5 pour 31, 10 pour un millier, 20 pour un million).
  Exemple dans la vie de tous les jours (dictionnaire).
  C'est de l'informatique parce que... 
  & Oral Collectif
  & (Aucun)
  \\
  \hline
\end{tabular}
\end{center}

\paragraph{Règles de la première partie}
Les élèves se répartissent en binômes (un(e) informaticien(ne), et un robot). 10 cartes sont mélangées et placées faces cachées sur la table. L'un des deux élèves doit les trier par ordre croissant. Pour ce faire il est libre de les arranger comme il le souhaite et peut demander à son camarade de comparer la valeur de deux cartes. Le robot (seul à pouvoir regarder les faces des cartes) indique alors quelle carte est la plus grande, sans lui donner les valeurs. Les rôles tournent après une réussite, ou si on n'avance pas.


\paragraph{Règles de la troisième partie}


Devant la classe, on essaye de leur faire trouver une carte parmi 7 en seulement 3 coups. On peut les faire réessayer une ou deux fois pour qu'il y ai au moins une réussite et un échec. L'objectif est qu'ils comprennent que ce n'est que de la chance.

Y a-t-il un moyen d'y arriver à coup sûr ? On fait venir une personne supplémentaire qui va bouger les cartes comme bon lui semble (avant qu'on annonce la carte à chercher) et sans rien communiquer à la personne qui cherche. L'objectif est qu'ils comprennent qu'il faut trier les cartes.

  


\paragraph{C'est de l'informatique parce que...}
Les ordinateurs contiennenent beaucoup d'informations, et on a envie d'y accéder rapidement. On les trie pour ça (comme pour la troisième activité et pour les dictionnaires).


\end{document}%
