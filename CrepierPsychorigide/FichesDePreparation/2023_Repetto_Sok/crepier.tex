 
\documentclass[main.tex]{subfiles}
\begin{document}
\begin{center}
\begin{tabular}{|c|>{\hsize=0.3\hsize\centering\arraybackslash}c|c|c|c|}
\hline
\multicolumn{5}{|c|}{
Repetto Guilhem \& Sok Chanattan} \\
\hline
\multicolumn{1}{|l|}{\textbf{Classe} : CM2} & \multicolumn{3}{|l|}{\textbf{Titre} : Le crêpier psychorigide} & \multicolumn{1}{|l|}{\textbf{Séance n°2}} \\
\hline
\multicolumn{5}{|l|}{\textbf{Compétences travaillées} : Tri avec contraites.} \\
\multicolumn{5}{|l|}{\textbf{Objectifs} : Introduire le raisonnement d'algorithmique avec le tri.} \\
\hline
\textbf{Activité} & \textbf{Script} & \textbf{Notes/Objectifs} & \textbf{Matériel} & \textbf{Temps} \\
\hline
\thead{Préparation\\de la classe} & \thead{Temps de marge pour préparer les îlots etc.} & \thead{A faire en amont du début de classe.} & \thead{} & \thead{-} \\
\hline
\thead{Présentation\\de l'activité} & \thead{Introduction devant la classe entière.} & \thead{On commence par présenter le\\problème vulgarisé en\\expliquant l'objectif.} & \thead{Crêpes} & \thead{5} \\
\hline
\thead{Explication\\avec exemple} & \thead{Les faire reformuler 4 fois.} & \thead{Pratique avec quelques exemples.} & \thead{} & \thead{3} \\
\hline
\thead{Distribution\\du matériel} & \thead{Parallélisable\\pendant les explications.} & \thead{} & \thead{} & \thead{2} \\
\hline
\thead{Activité} & \thead{Extension possible avec la\\préservation du côté des crêpes.} & \thead{Chaque ilot peut simuler\\l'exécution d'un algorithme\\entre au moins deux élèves.} & \thead{} & \thead{15} \\
\hline
\thead{Remise en commun} & \thead{Bilan sur la 1$^{\text{ère}}$ partie\\et la solution qui marche à tous les coups.} &  \thead{Attirer l'attention de tous les ilots\\et demander à des élèves d'expliquer\\et d'expliquer aux autres.\\Simuler l'algorithme avec un élève\\devant les autres et\\des élèves derrière lui au tableau.} &\thead{ } & \thead{5} \\
\hline
\thead{Remise au travail} & \thead{Extension avec la préservation\\de couleur des crêpes, le fait de piloter son voisin etc.} & \thead{} & \thead{} & \thead{15} \\
\hline
\thead{Conclusion} & \thead{Bilan et trace écrite.} & \thead{Explications récapitulatives.} & \thead{ } & \thead{10} \\
\hline
\thead{C'est de\\ l'informatique car\dots} & \thead{C'est un problème concret qui se généralise\\à un tri d'objets pour  rechercher vite des objets/fichiers.\\Un algo lent met 56 jours, un algo\\efficace met 1 seconde. Il faut aussi donner des instructions\\claires à l'ordinateur.} & \thead{} & \thead{} & \thead{5} \\
\hline
\multicolumn{4}{|l|}{} &
\multicolumn{1}{|c|}{\textbf{Total :} 55} \\
\hline
\end{tabular}

\end{center}
\end{document}
