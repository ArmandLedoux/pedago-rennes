\documentclass[a4paper,11pt]{article}%

\usepackage{fullpage}%
\usepackage{url}%
\usepackage[T1]{fontenc}%
\usepackage[utf8]{inputenc}%
\usepackage[main=francais]{babel}% % Adjust the main language

\usepackage{graphicx}%

\title{Activité "La couleur par les nombres"}%
\author{Dylan Bellier \and Jean Jouve}%
\date{}%

\begin{document}%

\maketitle%

\section{Résumé de l'activité}

\paragraph{Objectif}
Traduire des images en suite de nombres et inversement

\url{https://fr.wikipedia.org/wiki/Run-length_encoding} (s'il faut plus 
d'informations)

\paragraph{Compétence}
Compter, dessiner

\paragraph{Niveau}
CE2 et plus


\paragraph{Durée}
Environs 45 minutes.

\section{Déroulement}

\begin{center}
\begin{tabular}{|l|p{5.5em}|p{15em}|p{6em}|p{6em}|}
  \hline
    \textbf{Durée} 
  & \textbf{Phases} 
  & \textbf{Activité et consigne} 
  & \textbf{Organisation} 
  & \textbf{Matériel} 
  \\
  \hline
  \hline
  5' 
  & Présentation de l'activité 
  & "Aujourd'hui nous allons convertir des nombres en images noire et 
    blanche !" 
  & Oral Collectif 
  & (Aucun)
  \\
  \hline
  2/3' 
  & Recherche collective 
  & La classe recherche et propose des idées pour convertir une image en 
    nombres. Pour l'inspirer, nous dessinons une image dans un cadriage au 
    tableau.
  & Oral Collectif 
  & (Aucun)
  \\
  \hline
  2/3' 
  & Consignes 
  & Présentation d'une traduction simple avec une petite image, celle précedemment 
    dessinée.
  & Oral Collectif 
  & (Aucun)
  \\
  \hline
  10'/15'
  & Mise en pratique
  & Nous donnons quelques grilles avec les images codées en nombres à côté. 
    et des grilles vierges. Les enfants peuvent ainsi dessiner eux même des 
    images et les traduire en code afin de défier leurs amis.
  & En groupe de 2 ou 3 ou plus...
  & Grilles d'images à dessiner et grilles vierges.
  \\
  \hline
  5'
  & Ajout des couleurs
  & Comme au début de l'activité, la classe réfléchit sur comment coder les 
    couleurs puis nous leur donnons une solution. 
  & Oral Collectif
  & (Aucun)
  \\
  \hline
  10'
  & Ajout des couleurs
  & Les enfants peuvent maintenant mettre de la couleur dans leurs dessins.
  & En groupe de 2 ou 3 ou plus...
  & D'autres grilles vierges et crayons de couleurs
  \\
  \hline
  5'
  & Conclusion
  & Nous expliqueront en quoi l'activité que nous venont de pratiquer est de 
    l'informatique et lien avec l'activité sur la représentation des nombres 
    dans un ordinateur.
  & Oral collectif
  & (Aucun)
  \\
  \hline
\end{tabular}
\end{center}

\end{document}%