\documentclass[a4paper,11pt]{article}
 
\usepackage{fullpage}%
 
\usepackage[T1]{fontenc}%
\usepackage[utf8]{inputenc}%
 
\usepackage{mathpazo} 

\usepackage[main=francais,english]{babel}

\usepackage{graphicx}%
 
\usepackage{url}%
\usepackage{abstract}%
 
\usepackage{minted}%
 

 
\usepackage{xcolor}
\definecolor{very-light-gray}{gray}{0.97}
 

%%%%%%%%%%%%%%%%%%%%%%%%%%%%%%%%%%%
 
\parskip=0.5\baselineskip%
 
\sloppy%
 
%%%%%%%%%%%%%%%%%%%%%%%%%%%%%%%%%%%%%%%%%%%%%%%%%%%%%%%%%%%%%%%%%%%%%%
\begin{document} 
 
\begin{center}
\huge
Le crêpier psychorigide 
\end{center}

\textbf{Présentation~:}
Les élèves ont à leur disposition des rectangles de tailles différentes représentant des crêpes. Le crêpier, psychorigide, souhaite les ordonner par taille. Cela pose problème car il ne possède qu'une assiette et peut donc uniquement retourner des crêpes pour les trier. Il peut en retourner autant qu'il veut mais c'est la seule manipulation possible. Le but est de trier la pile de crêpe, tout d'abord en la regardant puis sans la regarder.

\textbf{\`A retenir~:}
Les élèves jouent deux rôles: celui qui donne les ordres et celui qui les exécute. La méthode suivie par celui qui exécute les ordres est appelée un algorithme. En informatique, celui qui donne les ordres est l'informaticien. Celui qui écoute et exécute est l'ordinateur. Les deux rôles sont bien distincts~: l'ordinateur n'a pas d'information sur les crêpes mis à part le fait qu'elles forment une pile.
\end{document}
