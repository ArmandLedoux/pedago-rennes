
\documentclass[a4paper,11pt]{article}
 
\usepackage{fullpage}%
 
\usepackage[T1]{fontenc}%
\usepackage[utf8]{inputenc}%
 
\usepackage{mathpazo} 
 
\usepackage[main=francais,english]{babel}
 
\usepackage{graphicx}%
 
\usepackage{url}%
\usepackage{abstract}%
 
\usepackage{minted}%

 
\usemintedstyle{borland}
 
 
\usepackage{xcolor}
\definecolor{very-light-gray}{gray}{0.97}
 

 
%%%%%%%%%%%%%%%%%%%%%%%%%%%%%%%%%%%
 
\parskip=0.5\baselineskip%
 
\sloppy%
 
%%%%%%%%%%%%%%%%%%%%%%%%%%%%%%%%%%%%%%%%%%%%%%%%%%%%%%%%%%%%%%%%%%%%%%
\begin{document} 
 
\begin{center}
\huge
Carrés magiques
\end{center}

\textbf{Présentation~:} L'activité est un tour de magie. On place des carrés de deux couleurs (une par face) pour former un plateau de 5 par 5. L'audience choisit la disposition des carrés. Le but du magicien est de retrouver un carré qui a été retourné par l'audience. L'assistant rajoute pour cela des carrés afin de faire un plateau de 6 par 6 constitué d'un nombre pair de carrés de chaque couleur par ligne et par colonne. Ainsi, le carré retourné change la parité de la ligne et la colonne sur lesquelles il se trouve.

\textbf{A retenir~:} Ce tour peut représenter une manière de détecter des erreurs. Détecter des erreurs est important lorsqu'on transmet des informations, car les données risquent d'être modifiées (si par exemple le matériel est défectueux). Ainsi les codes correcteurs d'erreurs permettent de résoudre ces problèmes. On peut retrouver une utilisation similaire avec les QR codes.
\end{document}
