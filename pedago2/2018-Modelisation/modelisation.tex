\documentclass{article}
\usepackage[utf8]{inputenc}
\usepackage[T1]{fontenc}
%\usepackage[french]{babel}

\title{Modélisation}
\author{Aurèle Barrière \& Thomas Mari \& Bastien Thomas}
\date{}

\begin{document}
\maketitle

Cette activité est distribuée sous licence CC-BY-SA.

\section{Introduction}
Le but de cet exercice est de faire comprendre qu'il ya plusieurs manières de représenter les mêmes données.
Et que certaines sont plus adaptées à résoudre certains problèmes.
Ici, on donne une description d'un chateau sous forme de texte, et on veut que les élèves se disent d'eux-mêmes qu'un dessin (graphe) serait plus approprié.
Nous présentons deux versions de l'exercice, pour différents niveaux d'aisance de lecture.
Certaines informations sont superflues. Le but de changer de représentation est aussi de s'initier à filtrer les informations importantes.
Quelques problèmes de graphes sont introduits.

\section{Contexte}
Un mauvais architecte nous a donné une description du château.

\section{Questions}
La première question à poser:\\
\textbf{La reine se trouve dans sa bibliothèque. Elle aimerait aller dans le grenier, comment peut-elle faire?}

Après quelques temps à essayer, les élèves ne vont pas trouver de solution (il n'y en a pas).
Au bout d'un moment, on leur rajoute l'information suivante:\\
\textbf{Dans la première version des plans, l'architecte avait oublié de préciser qu'il existe une porte entre la salle à manger et la cuisine.}

Si les élèves trouvent la solution, les questions suivantes peuvent être posées:\\
\textbf{Quel est le chemin le plus rapide?}\\
\textbf{La chambre du roi est fermée, comment peut-elle accéder rapidement au grenier?}

\newpage
\section{Description Du Château}
\begin{itemize}
\item Une porte relie la salle à manger et la chambre du roi.
\item Il y a une porte reliant la cuisine et la chambre des servants.
\item La chambre de la reine a une porte donnant sur le jardin.
\item Dans la cuisine, il y a un escalier qui mène au grenier.
\item Depuis la chambre des servants, on peut aller à la buanderie.
\item L'entrée de la tour de guet est dans le jardin.
\item Les chambres du roi et de la reine sont reliées par une porte.
\item La cuisine est reliée à la buanderie.
\item Depuis le Hall d'entrée, un couloir mène à la chambre du roi.
\item Une escalier relie les geôles et la chambre des servants.
\item La porte principale relie le hall d'entrée au jardin.
\item Dans la chambre de la reine, il y a un accès vers la bibliothèque.
\item Du hall d'entrée, on peut aller dans la salle à manger.
\item Dans le grenier, une trappe descend vers la cuisine.
\end{itemize}

\newpage
\section{Description difficile}

Le roi étant gourmand, et n'aimant guère marcher longtemps, une porte reliant sa chambre et la salle à manger s'imposait.
Pour préparer le déjeuner le matin, les servants doivent pouvoir se rendre directement à la cuisine. Ainsi, un couloir étroit relie la cuisine et la chambre des servants.
À la demande de la reine, s'extasiant à la vue d'une floraison printannière, sa chambre fût placée à côté du jardin, et une petite porte lui permet d'y accéder aussi souvent qu'elle le souhaite.
Attention, si vous empruntez l'escalier entre la cuisine et le grenier, la troisième marche est abimée.
Les servants se plaignent souvent des vapeurs émanant de la buanderie et traversant le petit escalier tortueux qui la relie à leur chambre.
La reine n'aime pas que les soldats traversent régulièrement le jardin pour entrer dans la tour de guet. Malheureusement, c'en est le seul accès.
Tous les soirs, la reine ouvre la porte qui sépare sa chambre de celle du roi, et lui souhaite une bonne nuit.
En attendant qu'un plat finisse de cuire, les servants peuvent passer immédiatement de la cuisine à la buanderie pour s'occuper d'autres tâches.
En rentrant d'une partie de chasse, le roi entre dans le jardin, franchit la grande porte pour passer par le magnifique Hall d'entrée, puis par la porte décorée qui le relie à sa chambre.
Les servants osent rarement ouvrir la petite porte de métal qui mène, depuis leur chambre, aux geôles sordides.
La reine aime se réfugier dans la lecture et le silence de sa bibliothèque. À sa demande, la bibliothèque n'est acessible que depuis sa chambre.
Le Hall d'entrée est vraiment impressionant. Vous êtes acceuillis par des statues imposantes des anciens rois, des plantes grimpantes de taille étonnante qui vous cachent presque la porte qui mène à la salle à manger.
Ce n'était peut-être pas une très bonne idée d'installer cette trappe dans le grenier qui donne sur la cuisine. Plus d'un servant est déjà tombé par mégarde.

\section{À ajouter}
Une version encore plus simple (moins de salles).
Problèmes de graphes: chemin hamiltonien (tour de garde), coloration.

\end{document}

