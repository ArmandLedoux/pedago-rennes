\documentclass[a4paper,10pt]{article}
\usepackage[utf8]{inputenc}
\usepackage{tikz}
\usepackage{amsmath}
\usepackage{fullpage}

%opening
\title{Exemple 4}
\date{}

\begin{document}

\maketitle

\section{Nombre de joueurs et situation initiale}
\begin{itemize}
	\item Rôles : l'arbitre, un attaquant, et deux joueurs, nommés joueur 1 et joueur 2.
	\item Au début, les joueurs 1 et 2 et l'attaquant possèdent chacun une clé triangulaire différente. Tous les joueurs possèdent toutes les clés carrées correspondantes. Tous les joueurs savent quelles clés possèdent les autres joueurs (laissez les visibles).
	\item But : le joueur 1 piochera une carte Secret, et devra l'envoyer au joueur 2 en s'assurant qu'elle reste secrète.
\end{itemize}




\section{Joueur 1}
\begin{enumerate}
	\item Piochez une carte Nombre. Mettez une copie de votre Identité et la carte Nombre (dans cet ordre) dans une boîte, fermez-la avec la clé carrée correspondant au joueur 2, et envoyez la boîte au joueur 2.
	\item Si vous recevez une boîte et qu'en l'ouvrant, vous trouvez dans cet ordre votre carte Nombre, et une deuxième carte Nombre : piochez une carte Secret, et mettez (dans cet ordre) la deuxième carte Nombre et la carte Secret dans une boîte, fermez la avec la clé carrée correspondant au joueur 2, et envoyez la au joueur 2.
\end{enumerate}

\section{Joueur 2}
\begin{enumerate}
	\item Si vous recevez une boîte que vous pouvez ouvrir avec votre clé triangulaire, ouvrez la. Elle doit contenir l'identité d'un joueur et une carte Nombre. Si c'est le cas, piochez une deuxième carte Nombre. Regardez la carte Identité que vous avez reçue, et prenez la clé carrée correspondante. Mettez la carte Nombre que vous avez reçue et celle que vous avez piochée (dans cet ordre) dans une boîte, et fermez la avec cette clé.
	\item Si vous recevez une boîte et qu'en l'ouvrant, vous trouvez dans cet ordre votre deuxième carte Nombre et une carte Secret, le protocole est terminé.
\end{enumerate}

\section{La faille}
Ce protocole contient une faille, moins intuitive que les autres, car elle suppose que le joueur 1 doit volontairement utiliser le protocole pour parler avec l'attaquant (sans savoir que celui-ci est malhonnête) :
\begin{itemize}
	\item Le joueur 1 envoie sa première boîte à l'attaquant, en l'ayant fermée avec la clé carrée de l'attaquant. Celui-ci peut l'ouvrir, et mettre son contenu dans une autre boîte qu'il ferme avec la clé carrée du joueur 2.
	\item Le joueur 2 trouvera la carte Identité du joueur 1 dans sa boîte, puisque l'attaquant n'a pas changé le contenu. Il va donc renvoyer deux cartes Nombres dans une boîte fermée avec la clé carrée du joueur 1 (l'attaquant ne peut donc pas l'ouvrir).
	\item Le joueur 1 va recevoir la boîte envoyée par le joueur 2, qui contient deux cartes Nombres, dont la sienne. Il va donc renvoyer la deuxième carte Nombre et une carte Secret à la personne avec qui il pense communiquer, c'est-à-dire l'attaquant.
	\item L'attaquant va recevoir une boîte contenant la carte Nombre piochée par le joueur 2, et une carte Secret. Il peut transférer son contenu dans une autre boîte qu'il ferme avec la clé du joueur 2. Il peut même changer le secret au passage.
\end{itemize}

\section{En pratique}
Dans cet exemple, le joueur 1 parle volontairement avec l'attaquant (sans savoir qu'il est malhonnête), mais le joueur 2 croit qu'il parle avec le joueur 1, alors que les messages qu'il reçoit viennent en réalité de l'attaquant.\\
Cette attaque a des applications réelles : imaginons une situation où le joueur 2 est le serveur d'une banque, le joueur 1 est un client de cette banque, et l'attaquant est une personne tierce qui se fait passer pour la banque auprès du joueur 1 (après une attaque par hameçonnage, par exemple). Le client va alors parler avec l'attaquant (sans savoir qu'il est malhonnête), et l'attaquant va de son côté communiquer avec la banque, qui croît parler avec son client. Lorsque l'attaquant reçoit le Secret choisi par le joueur 1, il peut décider de le changer, par exemple avec un autre Secret, qui dirait "transférez tout l'argent de mon compte vers cet autre compte". La banque croyant parler avec le client va alors exécuter l'instruction.


\section{Comment corriger l'attaque ?}
Le joueur 2 peut ajouter, dans la boîte qu'il envoie, une copie de sa carte Identité. Comme l'attaquant ne peut pas modifier le contenu de cette boîte (il ne peut pas l'ouvrir, car seul le joueur 1 pourra l'ouvrir), le joueur 1 recevra une boîte contenant la carte Identité du joueur 2, alors qu'il attend cette de l'attaquant. Il comprend alors ce qu'il se passe. De même, si l'attaquant crée lui-même une telle boîte, il doit piocher une carte Nombre comme s'il était le joueur 2 ; ce dernier, dans le dernier message qu'il recevra, s'attendra à trouver sa carte Nombre, et en trouvera une autre.
\end{document}
