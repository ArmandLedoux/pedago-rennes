\documentclass[a4paper,10pt]{article}
\usepackage[utf8]{inputenc}
\usepackage{tikz}
\usepackage{amsmath}
\usepackage{fullpage}

%opening
\title{Exemple 5: Protocoles de vote}
\date{}

\begin{document}

\maketitle

\section{Nombre de joueurs et situation initiale}
\begin{itemize}
	\item Rôles : l'arbitre, un serveur au moins 3 votants. L'attaquant peut être l'un des votants où une personne extérieure.
	\item Au début, le serveur possède une clé triangulaire et tous les votants possèdent un exemplaire de la clé carrée correspondante. Chaque votant possède une clé triangulaire différente, et le serveur possède toutes les clés carrées correspondantes.
	\item But : Le serveur organise deux votes. Les votants vont devoir communiquer avec le serveur pour transmettre leur vote pour la première élection. Puis le serveur annoncera les résultats. Puis le deuxième vote a lieu, et se passe de la même manière.
\end{itemize}


\section{Premier vote}
\subsection{Serveur}
	\begin{enumerate}
		\item Annoncez publiquement (en parlant) aux votants qu'ils vont devoir voter, en vous envoyant des messages contenant leur vote, pour choisir le plat qu'ils mangeront ce soir : qu'ils votent 1 pour des lasagnes, 2 pour le chou bouilli moisi depuis un mois, et 3 pour les algues au pétrole ramassée sur une plage de Brest.
		\item Ouvrez secrètement toutes les enveloppes que vous recevez (vous possédez normalement toutes les clés nécessaires). Chacune contient d'abord l'identité du votant, puis son vote. Vérifiez que tout le monde a voté, que chaque personne n'a voté qu'une seule fois, puis comptez les résultats et annoncez les.
	\end{enumerate}

\subsection{Votants}
	\begin{enumerate}
		\item Prenez une copie de votre carte identité et une carte Nombre portant le numéro correspondant à votre choix. Mettez les dans cet ordre dans une boîte que vous fermez avec votre clé \underline{triangulaire} (cela prouve que vous seul avez pu fermer cette boîte).
		\item Mettez cette boîte dans une autre boîte que vous fermez avec la clé carrée du serveur. Puis envoyez la boîte au serveur.
	\end{enumerate}

\section{Deuxième vote}
\subsection{Serveur}
	\begin{enumerate}
		\item Annoncez publiquement (en parlant) aux votants qu'ils vont devoir voter en vous envoyant des messages contenant leur vote pour se prononcer sur une loi sur le salaire des députés : qu'ils votent 1 pour doubler le salaire des députés, 2 pour laisser le salaire des députés tel qu'il est, et 3 pour diviser par deux le salaire des députés.
		\item Ouvrez secrètement toutes les enveloppes que vous recevez (vous possédez normalement toutes les clés nécessaires). Chacune contient d'abord l'identité du votant, puis son vote. Vérifiez que tout le monde a voté, que chaque personne n'a voté qu'une suele fois, puis comptez les résultats et annoncez-les.
	\end{enumerate}
\subsection{Votants}
	Procédez exactement comme lors du premier vote.

\section{La faille}
Ce protocole contient une faille, qui vient du fait que les messages des votes pour le premier et le deuxième vote ont exactement la même forme :
\begin{itemize}
	\item L'attaquant peut faire une copie de tous les messages utilisés pour le premier vote, tout en laissant celui-ci avoir lieu ; puis lors du deuxième vote, il intercepte tous les messages envoyés par les votants, et envoie à la place les copies de leur messages envoyés lors du premier vote.
	\item Le serveur, comme lors du premier vote, verra que chaque votant a voté une seule fois pour le deuxième vote, et ne détectera donc pas de problème.
	\item Statistiquement, les gens voteront presque tous 1 pour le premier vote. En procédant ainsi, l'attaquant peut faire croire au serveur que les votants ont presque tous voté 1 sur le deuxième vote, et ainsi faire accepter une loi pour doubler le salaire des députés.
\end{itemize}

\section{En pratique}
Même si les deux votes concernent des sujet différent, la méthode utilisée pour voter (le protocole, donc) est exactement la même sur les deux votes. L'attaquant peut ainsi réutiliser des messages envoyés pour le premier vote pour modifier le déroulement du deuxième vote. Ce type d'attaque s'appelle une \emph{attaque par rejeu}. Cela peut aussi être utilisé pour usurper l'identité de quelqu'un en se connectant à un serveur : si un serveur demande qu'on lui envoie notre identité et notre mot de passe, même chiffré, un attaquant peut récupérer ce message, et même s'il ne peut pas le déchiffrer, il peut l'envoyer au serveur, qui l'acceptera car l'identité et le mot de passe qu'il recevra correspondent. Il croira alors avoir identifié la personne désignée dans le message, ce qui est faux.


\section{Comment corriger l'attaque ?}
On peut corriger l'attaque en complexifiant le protocole, de sorte que les votants demandent d'abord l'autorisation de voter au serveur. Celui-ci peut alors leur envoyer une carte nombre, dans une boîte fermée avec leur clé carrée. Eux seuls sont capables de l'ouvrir. Ils peuvent alors inclure cette même carte Nombre dans la boîte contenant leur vote, pour prouver au serveur qu'ils sont bien la personne indiquée par la carte Identité dans l'enveloppe.
\end{document}
